\documentclass[10pt,a4paper]{scrartcl}
\usepackage[utf8]{inputenc}
\usepackage[T1]{fontenc}
\usepackage[french]{babel}
\frenchsetup{StandardItemLabels=true}
\usepackage{textcomp}
\usepackage{amsmath,amssymb}
\usepackage{lmodern}
\usepackage{graphbox}
\usepackage[dvipsnames,svgnames]{xcolor}
\usepackage{microtype}
\usepackage{hyperref}
\usepackage{lipsum}
\usepackage{showkeys} \hypersetup{colorlinks=true,linkcolor=Brown,breaklinks=true,bookmarks
=true,pdfstartview=XYZ}
\title{Équations différentielles}
\author{Gloria Faccanoni}
\date{15 janvier 2014}
\begin{document}
\maketitle
\lipsum[10]
\tableofcontents
\section*{Introduction}
\lipsum[1-2]
\section{Rappels}\label{rapp}
\lipsum[3]
\subsection{Condition Initiale}
\lipsum[4]\marginpar{SIUUUUUU}
\subsection{Problème de Cauchy}
\lipsum[5]
\section{Exercices}
\lipsum[6]
comme indiqué dans~\ref{rapp} alors test.
Google.\footnote{\href{http://www.google.com}{Google}}

\begin{enumerate}
\item pain
\item lait
\item farine
\end{enumerate}

\begin{description}
\item[Titre:]Edition scientifique avec \LaTeX
\item[Duree:]12h
\end{description}
\end{document}