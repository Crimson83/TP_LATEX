\documentclass[10pt,a4paper]{scrartcl}
\usepackage{textcomp}
\usepackage{mathtools,amssymb}
\usepackage{amsthm}
\theoremstyle{plain}
\newtheorem{theoreme}{Théorème}[section]
\newtheorem{proposition}[theoreme]{Proposition}
\newtheorem{corollaire}[theoreme]{Corollaire}
\newtheorem{lemme}[theoreme]{Lemme}
\newtheorem{exo}{Exercice}
\theoremstyle{definition}
\newtheorem{definition}[theoreme]{Définition}
\theoremstyle{remark}
\newtheorem*{remarque}{Remarque}
\usepackage{lmodern}
\usepackage{microtype}
\usepackage[dvipsnames,svgnames]{xcolor}
\usepackage{hyperref}\hypersetup{colorlinks=true,linkcolor=Brown,citecolor=
ForestGreen}
\usepackage{lipsum}
\begin{document}
\section{Rappels}
\begin{definition}
\lipsum[10][1-2]
\end{definition}
\begin{proposition}
\lipsum[1][1-2]
\end{proposition}
\begin{proof}
\lipsum[1][3-4]
Notons le petit carré automatiquement inséré pour signaler la fin de la preuve (il s'appelle QED, acronyme de ``quod erat demonstrandum'').
\end{proof}
\begin{corollaire}
\lipsum[1][5-6]
\end{corollaire}
\begin{exo}
\lipsum[1][7-8]
\end{exo}
\begin{exo}
\lipsum[1][9-10]
\end{exo}
\section{Approfondissements}
\begin{definition}
\lipsum[2][1-2]
\end{definition}
\begin{lemme}
\lipsum[2][3-4]
\end{lemme}
\begin{proof}
\lipsum[2][5-6]
\end{proof}
\begin{theoreme}
\lipsum[2][7-8]
\end{theoreme}
\begin{proof}
\lipsum[2][9-10]
\end{proof}
\begin{remarque}
\lipsum[2][11-12]
\end{remarque}
\begin{exo}
\lipsum[3][1-2]
\end{exo}
\end{document}